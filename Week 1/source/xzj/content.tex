\item Proof by introduction
\begin{proof}
We can easily prove the lemma using induction:
\begin{enumerate}
    \item Basic step\\
    When $\displaystyle n = 2^1$, the lemma obvirusly holds because
	\[\binom{2}{1} = 0\]
    \item Induction step\\
    Assume that the lemma holds where $n$ equals to $2^d$, namely
	\[\binom{n}{k} = 0\]
	when $1 \leq k \leq n - 1$.

	Then we will show that the lemma holds when $n$ equals to $2^{d + 1}$.

	According to the hints, on the one hand, the number of paths to the $n$-th line and the $k$-th grid equals to
	\[\binom{n}{k}\]
	On the other hand, we can also calculate the number by Multiplication Principle and Addition Principle, namely
	\begin{equation}\label{equation1}
	\binom{n}{k} = \sum_{i = 0}^{n / 2} \binom{n / 2}{i} * \binom{n / 2}{k - i}
	\end{equation}
	Then our proof divides into three cases:
	\begin{itemize}
		\item When $1 \leq k \leq n / 2 - 1$, it is obvirusly that
		\[\binom{n / 2}{k} = 0\]
		and
		\[\binom{n / 2}{0} = \binom{n / 2}{n / 2} = 1\]
		According to the equation \eqref{equation1}, we have
		\[\binom{n}{k} = \binom{n / 2}{k - 0} + \binom{n / 2}{k - n / 2}\]
		 By assumption, when $1 \leq k < n / 2$, we have
		\[\binom{n / 2}{k} = 0\]
		Notice that $k - n / 2 < n / 2$, so we have
		\[\binom{n / 2}{k - n / 2} = 0\]
		Therefore \[\displaystyle \binom{n}{k} = 0\]
		\item
		When $k = n / 2$, it is obvirusly that
		\[\binom{n / 2}{k} = \binom{n / 2}{k - n / 2} = 1\]
		Therefore
		\[\binom{n}{k} = (1 + 1) = 2 \equiv 0 \pmod{2}\]
		\item
		When $n / 2 \leq k < n$, it is easy to see that
		\[\binom{n / 2}{k} = 0\]
		And by assumption
		\[\binom{n / 2}{k - n / 2} = 0\]
		So that we have
		\[\binom{n}{k} = 0\]
	\end{itemize}

    \item Conclusion
	By induction, we have
    \[\binom{n}{k} \equiv 0 \pmod{2}\]
	where $1 \leq k \leq n - 1$ and $n = 2^d~(d \geq 1)$.
\end{enumerate}
\end{proof}
\item Direct Proof
\begin{proof}
We can denote the binomial coefficient as $2^p * q$, namely
\begin{equation}\label{equation2}
\binom{n}{k} = \frac{n!}{k!(n - k)!} = 2^p * q
\end{equation}
where $2 \nmid q$, $n = 2^d(d \geq 1)$ and $1\leq k\leq (n - 1)$.

We can easily calculate the exponent $p$ with the formular below
\begin{displaymath}
p = \sum_{i = 1}^{d}\left\lfloor \frac{n}{2^i} \right\rfloor - \sum_{i = 1}^{d}\left\lfloor \frac{k}{2^i}\right\rfloor - \sum_{i = 1}^{d}\left\lfloor \frac{n - k}{2^i}\right\rfloor
\end{displaymath}

Specially when $i = d$, we have
\begin{displaymath}
\left\lfloor \frac{n}{2^i} \right\rfloor = 1~\text{and}~\left\lfloor \frac{k}{2^i}\right\rfloor = \left\lfloor \frac{n - k}{2^i}\right\rfloor = 0
\end{displaymath}

According to the equation \eqref{equation2}, we have
\begin{displaymath}
	\begin{split}
		p &= 1 - 0 - 0 + \sum_{i = 1}^{d - 1} \left\lfloor \frac{n}{2^i}\right\rfloor - \left\lfloor \frac{k}{2^i}\right\rfloor - \left\lfloor \frac{n - k}{2^i}\right\rfloor\\
		&\geq 1  + \sum_{i = 1}^{d - 1} \frac{n}{2^i} - \frac{k}{2^i} - \frac{n - k}{2^i} = 1
	\end{split}
\end{displaymath}
Hence we have
\[2 \mid \binom{n}{k}\]
That is
\[\binom{n}{k} \equiv \pmod{2}\]
\end{proof}
