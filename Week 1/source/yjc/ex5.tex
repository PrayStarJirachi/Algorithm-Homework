We can easily come out an algorithm using the formular which is proved in Ex5. And here comes out of a theorem proved below.
\begin{theorem}
Let $p$ represents the maximum $2^d$ which is less than $n$, then either $m > n - p$ or $m - p \leq 0$ holds.
\end{theorem}
\begin{proof}
	It is obvirusly that $p \geq \dfrac{n}{2}$ according to the pseudocode and the definition. Then the proof divided into two cases:
	\begin{enumerate}
		\item If $m \leq \dfrac{n}{2}$, then $m - p \leq 0$;
		\item If $m > \dfrac{n}{2}$, then $m > \dfrac{n}{2} \geq n - p$.
	\end{enumerate}
\end{proof}

And with the theorem proved above, we can come up with a algorithm:
\begin{algorithm}[H]
	\caption{An efficient method calculating the binomial coefficient}
	\begin{algorithmic}[1]
		\Function {binom} {$n$, $m$, $p$}
			\If {$m > n$ \OR $m < 0$}
				\State \Return $0$
			\EndIf
			\If{$n < 2$} :
				\State \Return $1$
			\EndIf
			\If{$p = 0$} :
				\State $p = 1$
				\While{$p * 2 < n$}
					\State $p \leftarrow p * 2$
				\EndWhile
			\Else
				\While{$p \geq n$}
					\State $p \leftarrow p / 2$
				\EndWhile
			\EndIf
			\State \Return (\Call{Binom}{$n - p$, $m - p$, $p$} + \Call{Binom}{$n - p$, $m$, $p$}) $\bmod 2$
		\EndFunction
	\end{algorithmic}
\end{algorithm}

\noindent The implementation in python is displayed below
\pythoncode{source/xzj/ex5.py}
