Since we use dynamic programing, the runing time depends on how many binomial coefficients we calculate.
Consider the formula
\[\binom{n}{k}=\binom{n-1}{k-1}+\binom{n-1}{k}\]
It's easy to see that the area we calculate is an parallelogram in Pascal's triangle.
More specificly, in order to get the answer of $\binom{n}{k}$, there are exactly $n-k$ numbers we calculate in the columns from $1$ to $k$,
so we need to do $O(k\cdot (n-k))$ additions.\par
The time we need for a single addtion depends on the length of the result, namely $\log \binom{n}{k}$.
Since
\[\sum \binom{n}{k}=2^n\]
We can easily see that $\binom{n}{k}$ has a length up to $n$.
So the upper bound of the running time is $O(nk\cdot (n-k))$, which is not very efficient.
