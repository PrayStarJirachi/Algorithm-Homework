The additions we need is the same as Exercise2, $k\cdot (n-k+1)$.
But this time the addition can be done in $O(1)$ time because we only need to keep the result modulo $2$.
So the running time is $O(k\cdot (n-k+1))$.
The input size is $O(\log n)$ and the output size is $O(1)$. It is not an efficient algorithm comparing to the input/output size.
