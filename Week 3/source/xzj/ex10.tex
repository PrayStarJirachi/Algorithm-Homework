The two parts of the graph are independent, so we only need to caltulate the minimal spanning tree of the two parts respectively.
\begin{itemize}
	\item Left part:\par
		If we don't choose any of the parallel edges(containing three edges), we need to choose the other two edges.(1 choice).\par
		If we choose one of the parallel edges(3 choices), we only need to choose one edge from the other two edges(2 choices).\par
		In summary, we have $1 + 3 \times 2 = 7$ minimal spanning trees in the left part.
	\item Right part:\par
		Using the similar method, the right part have $1 + 2 \times 3 = 7$ minimal spanning trees in the right part.
	\item Combining two parts:\par
		Using Multiplication rule, we can get the total amount of the minimal spanning forests, namely $7 \times 7 = 49$
\end {itemize}
We also justify the answer by programming(enumerating every set of the edges, and check if it can form a minimal spanning forest):
\pythoncode{source/xzj/verify.py}
